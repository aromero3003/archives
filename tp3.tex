\documentclass[a4paper]{article}

\usepackage[utf8]{inputenc}
\usepackage[T1]{fontenc}
\usepackage[spanish]{babel}
\usepackage{amsmath, amssymb}
\usepackage{derivative}
\usepackage{tabularx}
\usepackage{siunitx}

\begin{document}

\makeatletter
\renewcommand{\@seccntformat}[1]{}
\makeatother
\section{PENDULO FISICO} % (fold)
\label{sec:pendulo_fisico}

\vspace{1cm}
\textbf{Grupo N°}: 06

\vspace{1cm}

\textbf{Integrantes}:
\begin{itemize}
    \item Gaspes Marcellini, Valentín
    \item Villafañe, Solange Agustina
    \item Romero Gonzales, Alan Xochtiel
    \item Lozano, Lourdes Nazarena
    \item Skurski, Dante Martín
    \item Colart Sosa, Luz Rosario
\end{itemize}

\vspace{2cm}
\begin{table}[h]
    \centering
    % \label{tab:data}
    \begin{tabularx}{1\textwidth}{
      | >{\centering\arraybackslash}X   | >{\centering\arraybackslash}X   | >{\centering\arraybackslash}X   | >{\centering\arraybackslash}X   | >{\centering\arraybackslash}X   | >{\centering\arraybackslash}X   | >{\centering\arraybackslash}X   | >{\centering\arraybackslash}X |
    }
        \hline
        \( l_{(\mbox{m})} \) & \( \Delta l _{(\mbox{m})}\) & \( m_{(\mbox{kg})} \) & \( \Delta m_{(\mbox{kg})} \) & \( M_{(\mbox{kg})} \) & \( \Delta M_{(\mbox{kg})} \) & \( \alpha \) & \( \Delta \alpha \) \\ \hline
                & & & & & & & \\ \hline
                % & & & & & & & \\ \hline
    \end{tabularx}
    \caption{Mediciones}
\end{table}

\subsection{Cálculos} % (fold)
\label{sub:calculos}
Altura de la masa al final de la colisión:
\[ \cos\alpha = \frac{l-h}{l} \]
\[ h = l - l\cos\alpha \]
\[ h = l(1-\cos\alpha) \]

Durante la colisión:
\[ \vec{P_o} = \vec{P_f} \]
\[ m \vec{V} = (m + M) \vec{V_1} \]

Después de la colisión, considerando rozamiento despreciable

\begin{align*}
    Ec_o &= Ec_f \\
    \frac{1}{2}(m+M)V_1^2 &= (m + M) g h \\
    \frac{1}{2} (m+M) \frac{m^2 V^2}{(m+M)^2} &= (m+M) gh \\
    V^2 &= \frac{2(m+M)^2 gh}{m^2} \\
    V &= \sqrt{\frac{2(m+M)^2 g [l(1+\cos\alpha)]}{m^2}} \\
    V &= \frac{m+M}{m} \sqrt{2gl(1+\cos \alpha)} \\
    V &= \left(1+ \frac{M}{m}\right) \sqrt{2gl(1+\cos \alpha)} \\
\end{align*}

% subsection Calculos (end)


\subsection{Propagación de errores} % (fold)
\label{sub:propagacion_de_errores}

Usando el método de derivadas parciales.
\begin{align*}
    \Delta V &= \left|\pdv{V}{l}\right| \Delta l + \left|\pdv{V}{m}\right| \Delta m + \left|\pdv{V}{M}\right| \Delta M + \left|\pdv{V}{\alpha}\right| \Delta \alpha \\
    \pdv{V}{l} &= \frac{m+M}{2 m \sqrt{l} } \sqrt{2g(1+\cos \alpha)} = \\
    \pdv{V}{m} &= -\frac{M}{m^2} \sqrt{2gl(1+\cos \alpha)} = \\
    \pdv{V}{M} &= \frac{\sqrt{2gl(1+\cos \alpha)}}{m} = \\
    \pdv{V}{\alpha} &= - \frac{m+M}{m} \frac{\sqrt{2gl} \sin\alpha}{2 \sqrt{1 + \cos\alpha}} = \\
    \Delta V &= \\
    V \pm \Delta V &= 
\end{align*}

% subsection Propagacion_de_errores (end)


% section pendulo fisico (end)

\end{document}
